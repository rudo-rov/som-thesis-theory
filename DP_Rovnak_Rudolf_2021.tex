% arara: xelatex
% arara: xelatex
% arara: xelatex


\documentclass[thesis=M,english]{FITthesis}[2019/12/23]

%\usepackage[utf8]{inputenc} % LaTeX source encoded as UTF-8
% \usepackage[latin2]{inputenc} % LaTeX source encoded as ISO-8859-2
% \usepackage[cp1250]{inputenc} % LaTeX source encoded as Windows-1250

% \usepackage{subfig} %subfigures
% \usepackage{amsmath} %advanced maths
% \usepackage{amssymb} %additional math symbols

\usepackage{dirtree} %directory tree visualisation
\usepackage{url}

% % list of acronyms
% \usepackage[acronym,nonumberlist,toc,numberedsection=autolabel]{glossaries}
% \iflanguage{czech}{\renewcommand*{\acronymname}{Seznam pou{\v z}it{\' y}ch zkratek}}{}
% \makeglossaries

\department{Department of Theoretical Computer Science}
\title{SimpleObjectMachine implementation}
\authorGN{Rudolf} %author's given name/names
\authorFN{Rovňák} %author's surname
\author{Rudolf Rovňák} %author's name without academic degrees
\authorWithDegrees{Bc. Rudolf Rovňák} %author's name with academic degrees
\supervisor{Ing. Petr Máj}
\acknowledgements{THANKS (remove entirely in case you do not with to thank anyone)}
\abstractEN{Summarize the contents and contribution of your work in a few sentences in English language.}
\abstractCS{V n{\v e}kolika v{\v e}t{\' a}ch shr{\v n}te obsah a p{\v r}{\' i}nos t{\' e}to pr{\' a}ce v {\v c}esk{\' e}m jazyce.}
\placeForDeclarationOfAuthenticity{Prague}
\keywordsCS{Replace with comma-separated list of keywords in Czech.}
\keywordsEN{Replace with comma-separated list of keywords in English.}
\declarationOfAuthenticityOption{1} %select as appropriate, according to the desired license (integer 1-6)
% \website{http://site.example/thesis} %optional thesis URL


\begin{document}

% \newacronym{CVUT}{{\v C}VUT}{{\v C}esk{\' e} vysok{\' e} u{\v c}en{\' i} technick{\' e} v Praze}
% \newacronym{FIT}{FIT}{Fakulta informa{\v c}n{\' i}ch technologi{\' i}}

\setsecnumdepth{part}
\chapter{Introduction}



\setsecnumdepth{all}
\chapter{State-of-the-art}

\chapter{Analysis and design}
\section{SOM design and features}
\begin{itemize}
	\item Data types: Integer, Char/String, (Boolean, Float)
	\item Basic arithmethics (boolean arithmethics?)
	\item Bitwise operations
	\item Classes - fields, methods, single inheritance, dynamic dispatch (late binding)
\end{itemize}

\section{Parsing}
TBD, use parser generator (ANTLR)?

\section{Interpretation}
Once the source code is parsed, the next step is executing it -- this step is called \textit{interpretation}. Interpretation is
As per \cite{wolczko-02-ast-interpret}, an interpreter for a language L can be defined as a mechanism for the direct execution of all programs
from L. It executes each element of the program without reference to other elements.

It is however very rare that any language is interpreted directly. In most cases of non-trivial languages, the interpretation process
is preceeded by parsing or compiling into some form of \textit{intermediate representation}. According to \cite{wolczko-02-ast-interpret},
this process removes lexical noise (comments, formating), elements can be abstracted/combined (into keywords, operations etc.) and reordered
into execution order (for example operators in an algebraic expression).

The choice of intermediate representation is therefore vital. It can determine a lot of aspects of interpretation - from the way of distributing
the interpreted program to time and space complexity of the interpreter.

\subsection{AST interpretation}
\textit{Abstract syntax tree (AST) is a tree representation of the source code of a computer program that conveys the structure of the source code.
Each node in the tree represents a construct occuring in the source code }\cite{deepsource-ast}.

As the name suggests, AST represents the source code in the form of a tree. During the transformation from the source code to AST, some information
is ommitted. Information that is vital for AST's according to \cite{deepsource-ast} is:
\begin{itemize}
	\item variables -- their types, location of their definition/declaration,
	\item order of commands/operations,
	\item components of operators and their position (for example left and right operands for a binary operator),
	\item identifiers and corresponding values.
\end{itemize}

\subsection{Bytecode interpretation}
Using a form of bytecode. Effective, requires:
\begin{itemize}
	\item designing the bytecode (instructions, bytecode file formats),
	\item AST to bytecode translation (AST -> bytecode instructions),
	\item actual bytecode interpretation.
\end{itemize}
Bytecode interpretation permits easier optimization.

\section{Optimization}
\begin{itemize}
	\item dead code elimination,
	\item constant propagation,
	\item others\ldots
\end{itemize}

\section{Virtual Machine}
Decide on memory hierarchy, garbage collection\ldots

\subsection{Garbage collection}


\chapter{Realisation}

\setsecnumdepth{part}
\chapter{Conclusion}


\bibliographystyle{iso690}
\bibliography{mybibliographyfile}

\setsecnumdepth{all}
\appendix

\chapter{Acronyms}
% \printglossaries
\begin{description}
	\item[AST] Abstract syntax tree
\end{description}


\chapter{Contents of enclosed CD}

%change appropriately

\begin{figure}
	\dirtree{%
		.1 readme.txt\DTcomment{the file with CD contents description}.
		.1 exe\DTcomment{the directory with executables}.
		.1 src\DTcomment{the directory of source codes}.
		.2 wbdcm\DTcomment{implementation sources}.
		.2 thesis\DTcomment{the directory of \LaTeX{} source codes of the thesis}.
		.1 text\DTcomment{the thesis text directory}.
		.2 thesis.pdf\DTcomment{the thesis text in PDF format}.
		.2 thesis.ps\DTcomment{the thesis text in PS format}.
	}
\end{figure}

\end{document}
